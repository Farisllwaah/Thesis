\chapter{Related Work}

\textbf{Workflow Management Systems} (WMS) such as Askalon \cite{Wieczorek2005}, Taverna \cite{Oinn2004}, and Pegasus \cite{Deelman2004} are designed to run scientific workflows on distributed environments. Many workflow systems use a particular workflow language or representation (BPEL \cite{BPEL}, SCUFL \cite{Oinn2004}, DAGMan \cite{Kalayci2010}, DAX \cite{Deelman2005}), which has a specification that can be composed by hand using a plain text editor. BPEL (Business Process Execution Language) \cite{BPEL} is the de facto standard for Web-service-based workflows with a number of implementations from corporates as well as open source organizations. DAGs (Directed Acyclic Graph) \cite{Deelman2005} is one of the task graph representations that are widely used as the programming model for many parallel applications because it is effective in expressing and optimizing irregular computations. Moreover, algorithms expressed as DAGs have the potential to alleviate the user from focusing on the architectural issues, while allowing the engine to extract the best performance from the underlying architecture. In this work, we extend the DAG model to be overhead aware, which assists us to model the process of task clustering in scientific workflows. 
Among these workflow management systems, Pegasus~\cite{Singh2008} has implemented and used a basic algorithm of task clustering called Horizontal Clustering (HC) that merges task at the same horizontal levels of the workflow. The clustering granularity (number of tasks within a cluster) of a clustered job is controlled by the user, who defines either the number of tasks per clustered job (\emph{clusters.size}), or the number of clustered jobs per horizontal level of the workflow (\emph{clusters.num}). We further extend HC to consider data transfer cost, the balancing between dependencies and computation, and the failure occurrence. 

\textbf{Workflow Scheduling}. There have been a considerable amount of work trying to solve workflow-mapping problem using DAG scheduling heuristics such as HEFT \cite{Topcuoglu2002}, Min-Min \cite{Blythe2005}, MaxMin \cite{Braun2001}, MCT \cite{Braun2001}, etc. Duan \cite{Rubing2005} and Wieczorek \cite{Wieczorek2005} have discussed the scheduling and partitioning scientific workflows in dynamic grids. The emergence of cloud computing \cite{Armbrust2009} has made it possible to easily lease large-scale homogeneous computing resources from commercial resource providers and guarantee the quality of services. In our case, since we assume resources are homogeneous, the resource selection is not a major concern and thus the scheduling problem can be simplified as a task clustering problem. 
More specifically, a plethora of  scheduling algorithms have been developed in the networking and operating system domains. Many of these schedulers have been extended to the hierarchical setting. Lifflander et al.~\cite{Lifflander2012} proposed to use work stealing and a hierarchical persistence-based rebalancing algorithm to address the imbalance problem in scheduling. Zheng et al.~\cite{Zheng2011} presented an automatic hierarchical load balancing method that overcomes the scalability challenges of centralized schemes and poor solutions of traditional distributed schemes. There are other scheduling algorithms~\cite{Braun2001} (e.g. list scheduling) that indirectly achieve load balancing of workflows through makespan minimization. However, the benefit that can be achieved through traditional scheduling optimization is limited by its complexity. The performance gain of task clustering is primarily determined by the ratio between system overheads and task runtime, which is more substantial in modern distributed systems such as Clouds and Grids. 




%Imbalance paper
\textbf{Overhead Analysis}~\cite{Ostberg2011, Prodan2008, Chen2011} is a topic of great interest in the Grid community. Stratan et al.~\cite{Stratan2008} evaluate in a real-world environment Grid workflow engines including DAGMan/Condor and Karajan/Globus. Their methodology focuses on five system characteristics: overhead, raw performance, stability, scalability, and reliability. They pointed out that head node consumption should not be negligible and the main bottleneck in a busy system is often the head node. Prodan et al.~\cite{Prodan2008} offered a complete Grid workflow overhead classification and a systematic measurement of overheads. In Chen et al.~\cite{Chen2011}, we extended~\cite{Prodan2008} by providing a measurement of major overheads imposed by workflow management systems and execution environments and analyzed how existing optimization techniques improve runtime by reducing or overlapping overheads. The prevalent existence of system overheads is an important reason why task clustering provides significant performance improvement for workflow-based applications. In this paper, we aim to further improve the performance of task clustering using such a classification of overheads. 

\textbf{Machine Learning in Workflow Optimization} has been used to predict execution time \cite{Rubing2009, 1015660, 1542747, da2013toward} and system overheads~\cite{Chen2011}, and develop probability distributions for transient failure characteristics. Duan et.al. \cite{Rubing2009} used Bayesian network to model and predict workflow task runtimes. The important attributes (such as the external load, arguments etc. ) are dynamically selected by the Bayesian network and fed into a radial basis function neural network to make further predictions. Ferreira da Silva et. al. \cite{da2013toward} used regression trees to dynamically estimate task behavior including process I/O, runtime, memory peak and disk usage. We reuse the knowledge gained from prior work on failure analysis, overhead analysis and task runtime analysis. We then use prior knowledge based Maximum Likelihood Estimation to integrate both the knowledge and runtime feedbacks and adjust the estimation accordingly. 



\textbf{Task Granularity Control} has also been addressed in scientific workflows. For instance, Singh et al.~\cite{Singh2008} proposed a level- and label-based clustering. In level-based clustering, tasks at the same level can be clustered together. The number of clusters or tasks per cluster are specified by the user. In the label-based clustering, the user labels tasks that should be clustered together. Although their work considers data dependencies between workflow levels, it is done manually by the users, which is prone to errors. Recently, Ferreira da Silva et al.~\cite{Ferreira-granularity-2013} proposed task grouping and ungrouping algorithms to control workflow task granularity in a non-clairvoyant and online context, where none or few characteristics about the application or resources are known in advance. Their work significantly reduced scheduling and queuing time overheads, but did not consider data dependencies. The low performance of \emph{fine-grained} tasks is a common problem in widely distributed platforms where the scheduling overhead and queuing times at resources are high, such as Grid and Cloud systems. Several works have addressed the control of task granularity of bags of tasks. For instance, Muthuvelu et al.~\cite{Muthuvelu2005} proposed a clustering algorithm that groups bags of tasks based on their runtime---tasks are grouped up to the resource capacity. Later, they extended their work~\cite{4493929} to determine task granularity based on task file size, CPU time, and resource constraints. Recently, they proposed an online scheduling algorithm~\cite{Muthuvelu2010,Muthuvelu2013} that groups tasks based on resource network utilization, user's budget, and application deadline. Ng et al.~\cite{Keat2006} and Ang et al.~\cite{Ang2009} introduced bandwidth in the scheduling framework to enhance the performance of task scheduling. Longer tasks are assigned to resources with better bandwidth. Liu and Liao~\cite{Liu2009} proposed an adaptive fine-grained job scheduling algorithm to group fine-grained tasks according to processing capacity and bandwidth of the current available resources. Although these techniques significantly reduce the impact of scheduling and queuing time overhead, they did not consider data dependencies.


\textbf{Workflow Patterns and Characteristics}~\cite{Yu2005a, Juve2013, Liu2008} are used to capture and abstract the common structure within a workflow and they give insights on designing new workflows and optimization methods. Yu and Buyya~\cite{Yu2005a} proposed a taxonomy that characterizes and classifies various approaches for building and executing workflows on Grids. They also provided a survey of several representative Grid workflow systems developed by various projects world-wide to demonstrate the comprehensiveness of the taxonomy. Juve et al.~\cite{Juve2013} provided a characterization of workflow from 6 scientific applications and obtained task-level performance metrics (I/O, CPU, and memory consumption). They also presented an execution profile for each workflow running at a typical scale and managed by the Pegasus workflow management system~\cite{Deelman2005}. Liu et al.~\cite{Liu2008} proposed a novel pattern based time-series forecasting strategy which utilizes a periodical sampling plan to build representative duration series. We illustrate the relationship between the workflow patterns (asymmetric or symmetric workflows) and the performance of our balancing algorithms. 
Some work in the literature has further attempted to define and model workflow characteristics with quantitative metrics. In~\cite{Ali2004}, the authors proposed a robustness metric for resource allocation in parallel and distributed systems and accordingly customized the definition of robustness. Tolosana et al.~\cite{Tolosana2011} defined a metric called Quality of Resilience to assess how resilient workflow enactment is likely to be in the presence of failures. Ma et al. ~\cite{Ma:2014:GDB:2560969.2561388} proposed a graph distance based metric for measuring the similarity between data oriented workflows with variable time constraints, where a formal structure called time dependency graph (TDG) is proposed and further used as representation model of workflows. Similarity comparison between two workflows can be reduced to computing the similarity between TDGs. In this paper, we focus on novel quantitative metrics that are able to demonstrate the imbalance problem in scientific workflows. 




\textbf{Workflow Partitioning} For convenience and cost-related reasons, scientists execute scientific workflows \cite{Bharathi2008, Rubing2005} in distributed large-scale computational environments such as multi-cluster grids, that is, grids comprising multiple independent execution sites. Topcuoglu \cite{Topcuoglu2002} presented a classification of widely used task scheduling approaches. Such scheduling solutions, however, cannot be applied directly to multi-cluster grids. First, the data transfer delay between multiple execution sites is more significant than that within an execution site and thus a hierarchical view of data transfer is necessary. Second, they do not consider the resource availability experienced in grids, which also makes accurate predictions of computation and communication costs difficult. Sonmez \cite{Sonmez2010} extended the traditional scheduling problem to multiple workflows on multi-cluster grids and presented a performance of a wide range of dynamic workflow scheduling policies in multi-cluster grids. Duan \cite{Rubing2005} and Wieczorek \cite{Wieczorek2005} have discussed the scheduling and partitioning scientific workflows in dynamic grids with challenges such as a broad set of unpredictable overheads and possible failures. Duan \cite{Rubing2005} then developed a distributed service-oriented Enactment Engine with a master-slave architecture for de-centralized coordination of scientific workflows. Kumar \cite{Kumar2002} proposed the use of graph partitioning for partition the resources of a distributed system, but not the workflow DAG, which means the resources are provisioned into different execution sites but the workflows are not partitioned at all. Dong \cite{Dong2007} and Kalayci \cite{Kalayci2010} have discussed the use of graph partitioning algorithms for the workflow DAG according to features of the workflow itself and the status of selected available resource clusters. Our work focuses on the workflow partitioning problem with resource constraints, in particular, the data storage constraint. Compared to Dong \cite{Dong2007} and Kalayci \cite{Kalayci2010}, we extend their work to estimate the overall runtime of sub-workflows and then schedule these sub-workflows based on the estimates. 



%Data Placement techniques try to strategically manage placement of data before or during the execution of a workflow. Kosar et al. \cite{Kosar2004} presented Stork, a scheduler for data placement activities on grids and proposed to make data placement activities as first class citizens in the Grid. In Stork, data placement is a job and is decoupled from computational jobs. Amer et al. \cite{Amer2012} studied the relationship between data placement services and workflow management systems for data-intensive applications. They proposed an asynchronous mode of data placement in which data placement operations are performed as data sets become available and according to the policies of the virtual organization and not according to the directives of the workflow management system (WMS). The WMS can however assist the placement services with the placement of data based on information collected during task executions and data transfers. Shankar \cite{Shankar2007} presented an architecture for Condor in which the input, output and executable files of jobs are cached on the local disks of the machines in a cluster. Caching can reduce the amount of pipelines and batch I/O that is transferred across the network. This in turn significantly reduces the response time for workflows with data-intensive workloads. In contrast, we mainly focus on the workflow partitioning problem but our work can be extended to consider the data placement strategies they have proposed in the future. 



%Park et al. \cite{Humphrey2008} limits the amount of parallel data transfer to avoid overloading supporting services such as data servers, which is called data throttling. Throttling is especially useful for unbalanced workflows in which one task might be idle while waiting for data to arrive. However, as discussed in \cite{Humphrey2008}, data throttling has an impact on the overall workflow performance depending on the ratio between computational and data transfer tasks. Therefore, performance analysis is necessary after the profiling of data transfers so that the relationship between computation and data transfers can be identified more explicitly. Rodríguez \cite{Rodríguez2012} proposed an automated and trace-based workflow structural analysis method for DAGs. Files transfers are accomplished as fast as the network bandwidth allows, and once transferred, the files are buffered/stored at their destination. To improve the use of network bandwidth and buffer/storage within a workflow, they adjusted the speeds of some data transfers and assured that tasks have all their input data arriving at the same time. Compared to our work, data throttling has a limit in performance gain by the amount of data transfer that can be reduced, while our partitioning approach can improve the overall workflow runtime and resource usage. 

%With the aim of dynamically balancing the computational load among resources, some jobs have to be moved from one resource to another and/or from one period of time to another, which is called task reallocation \cite{Tomas2012}. Caniou  \cite{Caniou2011} presented a reallocation mechanism that tunes parallel jobs each time they are submitted to the local resource manager (which implies also each time a job is migrated). They only needed to query batch schedulers with simple submission or cancellation requests.  Its authors also presented different reallocation algorithms and studied their behaviors in the case of a multi-cluster Grid environment. In \cite{Zhang2000}, a pre-emptive process migration method was proposed to dynamically migrate processes from overloaded nodes to lightly-loaded nodes. However, it can achieve a good balance only when there are some idle compute nodes (e.g. when the number of task processes is less than that of compute nodes). In our case with large scale scientific workflows, usually we have more tasks than available compute nodes . 

%Guo et al. \cite{Zhenhua2011} presented mechanisms to dynamically split and consolidate tasks to cope with load balancing and break through the concurrency limit resulting from fixed task granularity. They have proposed algorithms to address the load balancing problem in single-job systems and prior knowledge is not required. For multi-job cases, they used a shortest-job-first algorithm to minimize job turnaround time when combined with task splitting. Similarly, Ying et al. \cite{Ying2009} proposed a load-balancing algorithm based on collaborative task clustering. The algorithm divides the collaborative computing tasks into subtasks and then dynamically allocates them to the servers. Compared to these approaches, our work selects tasks to be merged based on their task runtime distribution and their data dependencies initially, without introducing additional overheads during the runtime. Also, a quantitative approach of imbalance measurement provides us a general view of multiple workflow instances with different runtime characteristics. 


%Tolerance Paper

\textbf{Failure Analysis and Modeling} \cite{Tang1990} presents system characteristics such as error and failure distribution and hazard rates. Schroeder et al. \cite{Schroeder2006} has studied the statistics of the data, including the root cause of failures, the mean time between failures, and the mean time to repair. Sahoo et al. \cite{Sahoo2004} analyzed the empirical and statistical properties of system errors and failures from a network of heterogeneous servers running a diverse workload. Oppenheimer et al. \cite{Oppenheimer2002} analyzed the causes of failures from three large-scale Internet services and the effectiveness of various techniques for preventing and mitigating service failure. McConnel \cite{McConnel1979} analyzed the transient errors in computer systems and showed that transient errors follow a Weibull distribution. In \cite{Sun2003, Iosup2008} Weibull distribution is one of the best fit for the workflow traces they used.  Based on these work, we measure the inter-arrival time of failures in a workflow and then provide methods to improve task clustering.  
More and more workflow management systems are taking fault tolerance into consideration. The Pegasus workflow management system \cite{Deelman2004} has incorporated a task-level monitoring system and used job retry to address the issue of task failures. They also used provenance data to track the failure records and analyzed the causes of failures \cite{Samak2011}. Plankensteiner et. al. \cite{plankensteiner2009fault} have surveyed the fault detection, prevention and recovery techniques in current grid workflow management systems such as ASKALON \cite{fahringer2007askalon}, Chemomentum \cite{schuller2008chemomentum}, Escogitare \cite{laforenza2007biological} and Triana \cite{taylor2007triana}. Recovery techniques such as replication, checkpointing, task resubmission  and task migration etc. have been provided. We are specifically joining the work of failure analysis and the optimization in task clustering. To be best of our knowledge, none of existing workflow management systems have provided such features. 




