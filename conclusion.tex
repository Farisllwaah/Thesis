\chapter{Conclusions}

\section{Sum of Contributions}

In chapter I, we have published papers

\section{Conclusions and Perspectives}

\section{Research Contributions}

The main contribution of this thesis is a framework for task clustering and workflow partitioning in distributed autonomous systems. Specially
\begin{enumerate}
\item We have developed an overhead aware workflow model to investigate the performance of task clustering in distributed environments. We present the overhead characteristics for a wide range of widely used workflows.
% In addition, we have showed how existing workflow optimization methods improve runtime performance by reducing some or all types of overheads.
\item We have developed partitioning algorithms that use heuristics to divide large-scale workflows into sub-workflows to satisfy resource constraints such as data storage constraint. 
\item We have built a statistical model to demonstrate that transient failures can have a significant impact on the runtime performance of scientific workflows. We have developed a Maximum Likelihood Estimation based parameter estimation approach to integrate both prior knowledge and runtime feedbacks. We have proposed fault tolerant clustering algorithms to dynamically adjust the task granularity and improve the runtime performance. 
\item We have examined the reasons that cause runtime imbalance and dependency imbalance in task clustering. We have proposed quantitative metrics to evaluate the severity of the two imbalance problems and a series of balanced clustering methods to address the load balance problem for five widely used scientific workflows. 
\item We have developed an innovative workflow simulator called WorkflowSim with the implementation of popular scheduling algorithms and task clustering algorithms. 
We have built an open source community for the users and developers of WorkflowSim. 
%\item Using a set of trace-based simulations, we compare the overall performance with existing approaches for a wide range of popular scientific workflows. We show that the proposed approach can provide significant improvement for the application. 
\end{enumerate}
 